\documentclass{book}
\title{Python入门与工程应用}
\author{汪昱帆}
\date{2022/12/4}
\usepackage[UTF8, scheme = plain]{ctex}
\usepackage{listings}
\usepackage{indentfirst}
\renewcommand*{\contentsname}{目\quad 录}
\renewcommand*{\chaptername}{}
\renewcommand{\chaptermark}[1]{\markboth{第\, \thechapter\, 章\ #1}{}}
\renewcommand{\sectionmark}[1]{\markright{\thesection\ #1}{}}
\setlength{\parindent}{2em}
\begin{document}
\maketitle
\tableofcontents
\setcounter{part}{-1}
\setcounter{chapter}{0}
\part{阅读建议}
本书涵盖了Python的基础、算法、工程方面的内容,涵盖面广


如果读者是Python的初学者,请先将语言基础读完


如果读者已经读完语言基础或不是Python的初学者,请按照自己的兴趣进行阅读,
等到有水平后,再进行字典式阅读。


希望读者多做实践,建议将书中所有代码全部上机实践

% \resizebox{\textwidth}{!}{

% }
\part{语言基础}
\chapter{初识Python}
\indent Python是一个编程语言,相信有些读者学过 \verb|C++| 等其他编程语言


Python的运用范围很广,可以用来做网络爬虫、科学计算、软件开发等


读者肯定用过软件,软件就是用编程语言做的

\section{编程语言与自然语言的区别}

\indent 自然语言不是很严格,比如:\verb|南京路到了|,这句话并不严谨,南京路究竟到哪了?但是编程语言十分严谨,比如,你如果忘写了一个右括号,计算机是看不出来的,程序就会无法运行


所以读者在编写程序时,一定要注意语法,
自然语言的语法并没有很严格的要求,
例如英语里将 \verb|like playing| 
写成了 \verb|like play| ,
在与外国人正常交流时,对方不会故意纠错,
但是将 \verb|print("Hello, world")| 
写成了 \verb|print["Hello, world"]| 
是会报错的

\section{Python的特点}

\begin{enumerate}
\item 优点
    \begin{itemize}   
        \item 十分简洁, C++ 需要用$100$行,Python可能$10$行就解决了
        \item 生态广, Pypi 里成千上万的包
        \item 资源多,Python已经有大量的教程,书籍
        \item ……
    \end{itemize}
\item 缺点
    \begin{itemize}
        \item 性能有待改进
        \item 不接近底层
        \item 不能直接以可执行文件发布
        \item ……
    \end{itemize}
\end{enumerate}

Python的环境配置在这里并不进行介绍,
请读者使用
搜索引擎搜索\verb|Python下载|


还建议读者搜索\verb|Pycharm|的下载配置,在以后做大型项目时,\verb|Pycharm|将成为开发的得力工具!
\chapter{第一个程序}
\section{代码}
\begin{verbatim}
# 示例2-1 Hello World
print("Hello, world")
\end{verbatim}
\footnote{\# 号是注释,后面的内容将不被Python识别,多行注释可以使用一对三引号,像""""""里就可以放多行注释}
\indent 请保存为\verb|HelloWorld.py|,双击运行,如果读者使用
的是Pycharm,请右键点击\verb|Run|

输出如下

\begin{verbatim}
Hello World
\end{verbatim}

其中,双引号可以换成单引号,双引号内可以填入任何内容

\begin{verbatim}
# 示例2-2 你好世界
print('你好世界')
\end{verbatim}

输出如下:
\begin{verbatim}
# 示例2-2 你好世界
print('你好世界')
\end{verbatim}

甚至可以输入一个数字
\begin{verbatim}
# 示例2-3 12345
print(12345)
\end{verbatim}

输出如下:
\begin{verbatim}
12345
\end{verbatim}

读者也可以尝试修改单引号内的文本

初学者在输入代码运行时,可能会犯以下错误导致无法运行

\begin{verbatim}
File "C:\Users\S.X.Y\Desktop\Python入门与工程应用\代码仓库\第二章 第一个程序\2-1 Hello World.py", line 1
    print("Hello World“)
                            ^
SyntaxError: EOL while scanning string literal

\end{verbatim}

或是
\begin{verbatim}
    File "C:\Users\S.X.Y\Desktop\Python入门与工程应用\代码仓库\第二章 第一个程序\2-1 Hello World.py", line 1
    print("Hello World")
                       ^
SyntaxError: invalid character ')' (U+FF09)
\end{verbatim}
等等问题


如果程序没有执行成功,请看\verb|2.3|

\section{第一个程序详细解析}
\indent 还记得\verb|2.1|的\verb|print(...)|么

\verb|print|在Python中是一个\verb|函数|

什么是函数呢?

相信读者学过数学当中的函数,比如
$f(x) = x^2$,则$f(5) = 25$,
Python中的函数与数学函数原理类似
\footnote{其实不管哪个编程语言的函数,原理都和数学函数类似}
\verb|print|是一个有参数的函数,就像$f(x)$中的参数$x$,在函数调用时,
必须传入参数,Python中参数是有类型的,

\verb|示例2-1|中,\verb|"Hello World"|是一个\verb|字符串|类型,
\verb|示例2-2|的\verb|'你好世界'|,也是字符串类型
,然而\verb|示例2-3|中,$12345$是一个整数类型,所以\verb|print|
是一个不限参数类型的函数

\verb|print|函数的作用是将参数的\textbf{值}
\footnote{假如参数是1+1,那么会打印2,因为1+1值为2}
原封不动的打印出来

留给读者一个思考题:假如说我键入代码\verb|print("1+1")|,
会出现什么有趣的结果呢?建议读者上机测试一下。这也是初学者最容易
错的地方

\section{Python基础语法规范}

\indent 如果说读者没有成功运行\verb|示例2-1|,
那么说明读者没有正确键入代码


请确保您已经切换到了英文半角输入法,否则Python将无法识别


例如\verb|print("“)|是不行的,这行代码出现
了两个中文符号——左括号以及右括号,会导致无法运行


其次,Python是大小写敏感的,例如,不可以将
\verb|print("ILOVEPYTHON")|写成
\verb|PRINT("ILOVEPYTHON")|,Python会不知道
PRINT函数到底是什么


\chapter{变量}
\indent 在玩游戏的时候,每一个任务都会有一个属性值


比如,血量,能力,等级等


这些都可以用变量表示


例如:变量血量值为100,能力值为100,等级值为3
\section{变量创建}
\indent 知道了变量的基本概念,我们来动手试着创建一个变量


变量的创建语法是:\verb|标识符 = 变量值|


注意,这个=\textbf{不是数学里的等号},这里的意思是赋值
数学里的等号则表示相等关系


比如,\verb|a = 2|是合法的,\verb|2 = a|是非法的


变量值可以是任何数据类型:可以是字符串,数字……


变量值也可以是变量,这种成为\verb|引用|


比如如下变量命名命令


\begin{verbatim}
alpha = 5
beta = alpha
\end{verbatim}


\verb|其中标识符|是有命名规则的,有\verb|强制|,就是语法层面的,有\verb|建议|,是为了让代码更易懂
\section{标识符强制命名规范}
标识符就是变量的名字,然而标识符是有命名规范的


有以下几点

\begin{itemize}
    \item 不能含有\verb|字母,下划线,中文,数字以外的字符|
    \item 开头不能为\verb|数字|
    \item 不能含有空格
\end{itemize}

你可能想问,中文也可以?没错!中文也可以,像代码\verb|你好世界 = "Hello World"|这样的代码
也是合法的,可运行的。并且,除非\textbf{开发国际项目}或者有特别的编码问题,笔者是鼓励读者使用
中文标识符的。
\section{标识符建议命名规范}
为了让代码的可读性\footnote{意思是让代码更容易让他人读懂}更上一层楼,一般要将代码写的简单易懂,除非说你的代码不给别人看


所以像\verb|a, b, c|这样的标识符是禁忌,好的标识符例如\verb|name, age|,一看就知道含义


以下是笔者总结出来的标识符命名规范:
\begin{itemize}
    \item 表达数据或函数时,将变量存储数据内容翻译成英文或缩写,小写写出,如有多个单词,用下划线隔开,例如\verb|alert_message|就是一个不错的标识符
    \item 表达类名时,将类的作用首字母大写,例如\verb|AlertMessage|
\end{itemize}

在项目中,最好遵守这两条规则,因为这是在计算机界公认的规则
\section{input函数}
\indent \verb|print|函数是用于将参数原封不动的输出,\verb|input|则相反,将输入的内容原封不动的赋值给变量


\verb|input|函数有一个参数,表示提示文字,还有一个返回值,会返回一个字符串,注意,是\textbf{字符串},
假如我输入\verb|1+1|,变量值为\verb|"1+1"|,注意,不是2,\verb|"1+1"|是字符串





做一个有趣的例子:
\begin{verbatim}
# 示例3-1 你的名字是什么?
name = input("你的名字是:")
print("原来你的名字是:", name)
\end{verbatim}
\footnote{print函数是可以输出多个参数的,比如print("Hello", "World")是合法的}


输出如下
\begin{verbatim}
你的名字是:小汪
原来你的名字是: 小汪
\end{verbatim}

至于为什么可以使用\verb|print(name)|这种句法,是因为\verb|name|已经成为标识符了,只要
Python看到\verb|name|,就会替换成\verb|小汪|


\section{变量修改}
\indent 变量是可以修改的


比如,我定义了变量\verb|a = 2|,我想把变量\verb|a|改为$3$,只需输入\verb|a = 3|


\begin{verbatim}
# 示例3-2 你就叫小汪!
name = input("你叫什么名字:")
print("原来你叫:",name)
name = "小汪"
print("我不管,你叫",name)
    
\end{verbatim}

在打印\verb|原来你叫:name|后,将\verb|name|改为\verb|小汪|,所以第四行代码输出:\verb|我不管,你叫小汪|


输出如下:
\begin{verbatim}
你叫什么名字:小钱
原来你叫: 小钱
我不管,你叫 小汪
\end{verbatim}
\section{删除变量}
\indent 当变量太多的时候,变量没有用是要删除空间释放内存的


删除使用命令\verb|del|,del是一个关键字\footnote{关键字就是Python预留好的“函数”,但使用语法各不相同,以后会接触30来个关键字}
语法是\verb|del 变量|,比如,删除\verb|name|变量,代码为\verb|del name|


这里给大家一个示例


\begin{verbatim}
>>> # 示例3-3 变量的消失.txt
>>> name = '小汪'
>>> print(name)
小汪
>>> del name # 删除变量name
>>> print(name)
Traceback (most recent call last):
  File "<pyshell#3>", line 1, in <module>
    print(name)
NameError: name 'name' is not defined
\end{verbatim}

\verb|NameError|意为找不到变量。\verb|name|已被删除,当然找不到。

变量删除我就不再赘述,在一般应用场景中,我们可能不怎么会用到变量删除

\section{示例-游戏属性打印}
\indent 这一节我们将制作第一个项目——游戏属性打印


这是一个很简单的项目,希望大家将代码在文件中敲一遍,运行一遍。


\subsection{模型解析}
\indent 不管多么简单的项目,都需要对其策略进行解析


游戏属性如下:
\begin{itemize}
    \item 血量
    \item 攻击能力
    \item 等级
\end{itemize}
这三个变量都是数字


变量分别命名为:\verb|blood, attack, grade|


模型已经分析完毕,接下来将开始代码制作
\subsection{变量读取}
\indent 接着就要开始读取变量了


前面说过,可以使用\verb|input|函数,接下来会使用\verb|input|函数读取变量


\begin{verbatim}
blood = input("输入血量:") # input可以有提示信息
attack = input("输入攻击能力:")
grade = input("输入等级:")
\end{verbatim}

现在已经可以读取变量了!


\subsection{变量输出}

\indent 读取完了变量,要输出


这里使用\verb|print|函数,前文说过\verb|print|函数可以有多个值,会分别输出


接下来是打印部分代码
\begin{verbatim}
print("血量为:", blood)
print("攻击能力为:", attack)
print("等级为:", grade)
\end{verbatim}

\subsection{完整代码}
\indent 到这一步,我们已经完成了游戏属性打印项目


完整代码:
\begin{verbatim}
# 示例3-3 游戏属性打印
blood = input("输入血量:") # input可以有提示信息
attack = input("输入攻击能力:")
grade = input("输入等级:")
print("血量为:", blood)
print("攻击能力为:", attack)
print("等级为:", grade)
\end{verbatim}

输出如下:
\begin{verbatim}
输入血量:100
输入攻击能力:20
输入等级:3
血量为: 100
攻击能力为: 20
等级为: 3
\end{verbatim}
恭喜你,你已经完成了第一个项目!
\chapter{基本数据类型}
\indent 之前笔者简单介绍了\verb|数字|以及\verb|字符串|,Python
肯定不止这些数据类型,还有\verb|浮点型|,\verb|复数型|,\verb|复合类型|
等


这一章为读者介绍一下Python内置的数据类型
\section{整数型}
\indent 整数型就是之前说的\verb|数字型|,在Python里叫\verb|int|类型
整数型可以直接使用数字表示,也可以使用\verb|int|类来表示


在\verb|示例3-3|中,笔者使用了3个\verb|int|类型的数据,所以,
\verb|int|型可以这样表示


\begin{verbatim}
alpha = 2
\end{verbatim}
或者,也可以使用\verb|int|类\footnote{类的实例化行为有点像函数,后面会详细介绍类}
\begin{verbatim}
alpha = int(2)
\end{verbatim}
上面的两行代码效果等价


或者,也可以传入其他类型作为参数,这属于\textbf{强制类型转换}
\section{浮点型}
\indent 浮点型,其实就是小数,在Python中,浮点型照样有两种表达形式,
浮点型称为\verb|float|



比如如下代码:
\begin{verbatim}
beta = 2.0
\end{verbatim}
\begin{verbatim}
beta = float(2)
# 会转换为float型
\end{verbatim}

\subsection{浮点数的易错点}
\indent 在有些考试中,可能会考你这样的题目
\begin{verbatim}
beta = float(2)
beta = ?
A.0     B.2     C.2.0   D.2.00
\end{verbatim}
不少同学会选择\verb|B|,\textbf{但是,编程语言中,float型不和int型相等}
除非说是编程题或者其他类型的题目,正确答案应该是\verb|C|
\subsection{杂谈-无穷大}
\indent 很多编程语言中,都没有无穷大
这种数据类型,但是\verb|Python|提供了!
我们可以通过\verb|float| API去使用这种
数据类型


表示方法为:\verb|float("inf")|,注意:


一定要带入float类,不可以写成\verb|"inf"|
或者\verb|"inf"|


以下是一个无穷大类型的使用示例


\begin{verbatim}
>>> alpha = float("inf")
>>> print(alpha)
inf
>>> print(alpha + 2.0)
inf
>>> alpha - alpha
nan
>>> alpha - 2.0
inf
    
\end{verbatim}
这基本符合了数学里无穷大的定义
\section{字符型}
\indent 这里终于可以畅聊字符串了


字符串也可以使用\verb|str|表示


比如:
\begin{verbatim}
alpha = "ILOVEPYTHON"
alpha = 'ILOVEPYTHON'
alpha = str("ILOVEPYTHON")
alpha = str('ILOVEPYTHON')
# 以上四行代码效果等价
\end{verbatim}
\subsection{易错点}
\indent 这是\textbf{所有计算机生可能犯过的错误}


比如以下题目:
\begin{verbatim}
以下代码:
print("1+1")
输出结果是:?
A.2   B.1+1     C.3     D.0
\end{verbatim}
很多人都会选\verb|B|,但是,字符串只会最原生态的存储信息,就算是一个式子\verb|"1+1"|,也会打印出\verb|1+1|,只有去掉
字符串,变成$1+1$,成为\textbf{表达式},才会计算结果输出$2$


字符串其实也不是完全原生态存储信息,会有几个\textbf{特殊}符号,会在下一节介绍


\subsection{特殊符号}
\indent Python中有很多特殊符号,比如\verb|\r|换行,\verb|\b|推格等,特殊符号会在此节介绍

\subsubsection{换行符n}
\indent 换行符\verb|\n|可以实现换行功能


使用方法如下


\begin{verbatim}
>>> print('这是换\n行\n符\n')
这是换
行
符
\end{verbatim}

\subsubsection{换行符r}

\indent 换行符\verb|\r|也是换行符,但会回到行首,但请注意,这个换行符只在Windows操作系统下有效。


使用方法如下:


源代码:

\begin{verbatim}
print("a\rb")
\end{verbatim}

输出:
\begin{verbatim}
b
\end{verbatim}

需要注意的是,\verb|\r|换行符在Python的解释器中不会起到任何作用

\subsection{原始字符串r}
\indent 这个不是换行符,使用方法是在字符串前加一个\verb|r|,像\verb|r'Hello Wang'|就是原始字符串,
在文件读取的时候十分有用,因为文件路径中的反斜杠会使Python报错,加上\verb|r|以后,就不会有这种情况
出现,这里不会很详细的介绍原始字符串,以后会慢慢接触。
\section{复数型}
\indent 在科学计算中,经常要用到复数,很多语言都没有提供复数,Python却提供了。

复数在数学的表达方式为 $a + bi$,其中 $a$ 是实部,$b$ 是虚部,$i$是虚数单位


复数这里也不做详细介绍


\begin{verbatim}
>>> 3+9j # 实部为3,虚部为9
(3+9j)
>>> (3 + 9j) + (9 + 3j)
(12+12j)
\end{verbatim}
\chapter{计算}
\indent 计算是编程语言最重要的部分之一
,Python几乎各个数据类型都可以完成计算,
Python还有二进制的位运算。本章将为读者介绍
Python中的运算
\section{整数和浮点型的运算}
\indent 之前和读者提及的数字型,严格意义上讲叫整数,生活中经常提及到的小数,在Python叫浮点数
\subsection{加减整除}
\indent 加减乘除可以说最简单的运算了,不过Python的浮点数与整数运算有一点坑。


Python的加减乘除就是用加号减号乘号除号完成运算,例如一下案例
\begin{verbatim}
>>> 3 + 1
4
>>> 5 - 2
3
>>> 11 * 11
121
>>> 11 / 11
1.0
\end{verbatim}
需要注意,Python中除法得出的永远是浮点数,整数和浮点数的运算也会得出类似结果
\begin{verbatim}
>>> 3.0 + 1
4.0
>>> 5.0 - 2
3.0
>>> 11 * 11.0
121.0
>>> 11.0 / 11
1.0
\end{verbatim}

Python的除法提供了整除这种运算符,将在下一节介绍。
\subsection{取余与整除}
\indent 整除其实就是将除法所得结果的小数部分舍去,如以下案例


\begin{verbatim}
>>> 3 // 2
1
>>> 5 // 2
2
\end{verbatim}
但需要注意,整除对浮点数是无效的

\begin{verbatim}
>>> 3.0 // 2
1.5
\end{verbatim}


取余的代码如下:
\begin{verbatim}
>>> 10 % 3
1
\end{verbatim}

取余就等于数学里的 $mod$ 算子
\subsection{自增运算符}
\indent 前面讲了变量,变量可以赋值,可以计算,变量还可以自赋值


比如,有一个变量\verb|wang|,它的值为$1.1$,自赋值以后,它就可以通过自赋值改变自身值


\begin{verbatim}
>>> wang = 1.1
>>> wang = wang + 1
>>> print(wang)
2.1
\end{verbatim}

从示例可以看到,\verb|wang|加了$1$后赋给了\verb|wang|,Python有这种专门的运算符叫\textbf{自增运算符}

\begin{verbatim}
>>> alpha = 1
>>> beta = 1
>>> alpha = alpha + 1
>>> print(alpha)
2
>>> beta += 1
>>> print(beta)
2
\end{verbatim}
由此可见,像\verb|+=|这种运算符会计算自身改变自身值,还有\verb|-=, *=, /=|

\begin{verbatim}
>>> alpha = 10
>>> alpha += 1
>>> print(alpha)
11
>>> alpha -= 1
>>> print(alpha)
10
>>> alpha *= 10
>>> print(alpha)
100
>>> alpha /= 100
>>> print(alpha)
1.0
\end{verbatim}
Python所有的运算符都支持这种自赋值运算。比如,下一节介绍的位运算,就会有与运算符\verb|&|,
与运算符的自赋值符号就是\verb|&=|
\subsection{位运算}
\indent 位运算是二进制层面的,所以在进行位运算时,Python会把输入的数转成二进制数,处理完再按原来
的进制输出。


位运算有与运算、或运算、异或运算、取反运算、左移运算、右移运算。


\subsubsection{与运算}

与运算是将两个操作数的每一位进行判断,如果都为1,对应的哪一位返回1,否则返回0


比如:$81$ 与 $64$,$81$ 二进制数为 $(1010001)_2$,$64$ 二进制数为 $(1000000)_2$,
得到了二进制数,就可以进行与运算了


需要注意的是,如果位数不够,在前面补上零。\\



\begin{tabular}{|c|c|c|c|c|c|c|}
\hline
1 & 0 & 1 & 0 & 0 & 0 & 1\\\hline
1 & 0 & 0 & 0 & 0 & 0 & 0\\\hline
1 & 0 & 0 & 0 & 0 & 0 & 0\\\hline
\end{tabular}


再将结果转换成十进制数,得出$64$


与的符号是\verb|&|,代码实现如下


\begin{verbatim}
>>> print(81 & 64)
64
\end{verbatim}

与与运算相对应的是或运算

\subsubsection{或运算}
\indent 以上述的运算为例,这一次计算 $81$ 或 $64$


同理,或运算需要将操作数转换成二进制数,如果这两个数的有一个是 $1$ 得出 $1$,否则得出$0$


直接开始计算\\
\begin{tabular}{|c|c|c|c|c|c|c|}
\hline
1 & 0 & 1 & 0 & 0 & 0 & 1\\\hline
1 & 0 & 0 & 0 & 0 & 0 & 0\\\hline
1 & 0 & 1 & 0 & 0 & 0 & 1\\\hline
\end{tabular}


得出结果$81$


或运算的符号为|,代码实现如下:

\begin{verbatim}
>>> print(81 | 64)
81
\end{verbatim}

\subsubsection{异或运算}


\indent 异或运算指的是两个二进制操作数,
如果两数相同,返回$0$,否则返回$1$。


比如 $81$ 异或 $64$\\

\begin{tabular}{|c|c|c|c|c|c|c|}
\hline
1 & 0 & 1 & 0 & 0 & 0 & 1\\\hline
1 & 0 & 0 & 0 & 0 & 0 & 0\\\hline
0 & 0 & 1 & 0 & 0 & 0 & 1\\\hline
\end{tabular}


$(10001)_2$十进制数为$17$,得出结果$17$


异或的运算符为\verb|^|


代码如下:
\begin{verbatim}
>>> print(81 ^ 64)
17
\end{verbatim}


\subsubsection{取反运算}
\indent 取反运算是取一个数的反码,再取补码
,这个比较复杂,而且取反运算是一种自反
运算\footnote{自反运算指的是其逆运算就是本身
,比如取反取反27与27是相同的,也就是说取反两次
会得到数字本身},而且它只有一个操作数,使用取反
只需要记住一个最基础的公式就可以了。


取反的符号是\verb|~|,使用方法与之前的不同
,是\verb|~a|,$~a = -(a+1)$,比如
$~81$就等于$-82$


代码如下:
\begin{verbatim}
>>> print(~27)
-28
\end{verbatim}

\subsubsection{左移运算}
\indent 左移的概念比较复杂,每左移一位
,就是乘 $(100)_2$,相当于每左移一位,就是乘2


左移符号为$>>$,比如$2 >> 2$就等于$8$,因为
$2 \times 2 \times 2$等于8


所以左移公式如下


\begin{equation}
a >> b = a \times 2 ^ b
\end{equation}


接下来看代码实例

\begin{verbatim}
>>> print(6 >> 2)
24
\end{verbatim}
\subsubsection{右移运算}
\indent 右移运算和左移相反,每左移一位,相当于除以$(100)_2$
,就是$2$ 。


注意,与普通除法不同,右移相当于整除运算。


右移符号为$<<$,比如$2 << 1$就是$1$,因为$\frac{2}{2} = 1$


所以右移公式如下


\begin{equation}
a << b = \frac{a}{2^b}
\end{equation}


接下来看代码示例
\begin{verbatim}
>>> print(8 >> 2)
2
\end{verbatim}

\subsection{运算表达式}
\indent 表达式其实就是数学里的综合算式,根据类型分为逻辑表达式以及算术表达式。


逻辑表达式会在后面介绍,算术表达式其实读者已经有所接触,只是之前碰到的都是只有一个
单项式的简单表达式,现实中表达式可能会很长很长


比如这是一个公式:
\begin{equation}
R_s = {2GM}{c^2}
\end{equation}


这是史瓦西半径的计算公式,但是这种算式代码是完全识别不了的,对应的
算术表达式如下:

\begin{verbatim}
R_s = 2 * G * M / (c * c)
\end{verbatim}


有几点需要注意:
\begin{itemize}
    \item 在算术表达式中,公式中的分数要写成除号\verb|/|
    \item 算术表达式不能省略乘号
    \item 算术表达式有严格的运算优先级,运算优先级见下一节
    \item 根号之类的运算符要用函数math.sqrt替代
\end{itemize}

既然说道\verb|math.sqrt|了,那就浅谈一下\verb|math.sqrt|

\subsubsection{sqrt}
\indent Python中的根号不可能像数学一样使用符号替代,所以在Python的库里,有一个叫\verb|math|的
库


库可以理解为工具箱,库里有很多函数和类,函数和类就是工具。


在库\verb|math|里有一个函数叫做\verb|sqrt|,它有一个参数:浮点型或者整数型


然而工具箱是要自取的,库的导入使用\verb|import|关键字“自取”,比如以下代码就可以导入库

\begin{verbatim}
import math
\end{verbatim}

使用库里的函数要写\verb|库名.函数名(参数)|


比如,我想计算$\sqrt{2}$,就可以使用以下代码

\begin{verbatim}
>>> import math
>>> 
>>> result = math.sqrt(2)
>>> print(result)
\end{verbatim}

读者可能注意到,在导入库后面笔者加了一个空行,这是因为\verb|代码规范|,在大型项目中,这种
规范非常重要,后面会讲解。
\subsubsection{运算优先级}
运算优先级表\footnote{此表的顺序是上高下底}如下:

\begin{tabular}{|c|c|c|}
运算符说明 & 运算符 & 优先级
小括号     & ()    & 19
索引      &  a[num]& 18
属性访问   &x.attribute & 17
乘方       & **    & 16
取反       & ~     & 15
符号运算   & +、-\footnote{需注意,这两个符号是正负,不是加减!} & 14
乘除       & *、/、//、\verb|%| & 13
加减       & +、-   & 12
位移       & <<、>> & 11
位与       & \verb|&| & 10
异或       & ^      & 9
位或       & |      & 8
逻辑运算符  & ==、!=、>、<、>=、<= & 7
is         & is、is not & 6
成员运算符  & in、not in & 5
逻辑非      & not        & 4
逻辑与      & and        & 3
逻辑或      & or         & 2
逗号        & par1, par2 & 1
 
\end{tabular}

请注意,这张表不需要刻意记。因为有些运算符都不怎么用,有些运算符不是算术运算符,此外,里面的绝大多数
运算符后面都会讲解。
\section{字符之间的计算}
\indent 其实Python的字符是支持运算的!


与其他语言不同,Python的字符支持加、乘。
\subsection{字符的加法}
\indent 字符的加法就是拼接两个字符


\begin{verbatim}
>>> str1 = 'abc' # 字符1
>>> str2 = 'def' # 字符2
>>> str3 = str1 + str2 # 拼接字符
>>> print(str3)
abcdef
\end{verbatim}
\subsection{字符的乘法}
\indent 字符的乘法其实就是多次叠放字符,代码实例如下:

\begin{verbatim}
>>> str1 = 'abc' # 字符
>>> num = 5      # 重复次数
>>> print(str1 * num)
abcabcabcabcabc
\end{verbatim}

\subsection{强制类型转换}
\indent 有时候,我们需要将一个字符型转换成浮点型或整数型,这时候可以借助\verb|强制类型转换|,
此时,类型标识符的行为更像函数\footnote{其实也不是函数,是一种较为复杂的数据结构:类}。它会返
一个值,表示转换后的值,代码实例如下:

\begin{verbatim}
>>> str1 = '50.0'
>>> float(str1) # 转换成浮点型
50.0
>>> type(float(str1)) # 数据类型判断
<class 'float'>
>>> type(str1)
<class 'str'>
>>> int(float(str1)) # 需要注意,字符型不能直接转换成整数型
50
>>> type(int(float(str1)))
<class 'int'>
>>> str(2.0)
2.0
>>> str(2)
2
>>> type(str(2))
<class 'int'>
\end{verbatim}
\chapter{复合类型}
\indent 与数据类型不同,复合类型是数据结构。一个数据结构可以存放大量数据。在不同语言中,数据结构
的要求也大不相同,比如,C++中,一个数据结构只能存放一种数据类型,Java也是如此。在不同语言中,数据
结构的称呼也大不相同,C++有映射、数组、集合,Python有元组、列表、字典、集合……


当然,大部分语言的数据结构的底层原理是相同的。比如映射和字典都是基于散列表\footnote{有些人习惯叫成哈希表}。


不同的数据结构也有各自的优点与缺点。比如映射和字典的内存占用过高,但是操作时间很短。
\section{列表}
\indent 列表可以理解为其他语言的数组。
\subsection{列表声明}
\subsection{列表添加}
\subsection{列表删除}
\subsection{成员运算}
\subsection{列表拼接}
\section{元组}
\subsection{元组的声明}
\subsection{元组拆包}
\subsection{元组的成员运算与拼接}
\section{列表与元组的切片}
\section{集合}
\subsection{集合的声明}
\subsection{集合的运算}
\subsection{利用集合的特性做个实例}
\section{字典}
\subsection{字典的声明}
\subsection{字典的添加}
\subsection{字典的取值}
\subsection{获取字典的键值表}
\subsection{字典的代价}
\chapter{分支结构}
\section{if,elif,else}
\subsection{逻辑表达式}
\subsection{if}
\subsection{elif}
\subsection{else}
\chapter{循环结构}
\section{while}
\subsection{死循环}
\section{for}
\subsection{循环语句}
\section{break与continue}
\chapter{代码复用}
\section{函数}
\subsection{不含参函数}
\subsection{含参函数}
\subsection{缺省参数}
\subsection{关键字参数}
\subsection{字典参数}
\section{类}
\subsection{类的声明}
\subsection{类的实例化}
\subsection{初始化特殊方法}
\subsection{类的使用}
\subsection{特殊方法表}
\section{映射与过滤函数}
\section{模块}
\part{算法}
\chapter{复杂度}
\section{浅谈大O复杂度}
\section{时间复杂度}
\section{空间复杂度}
\section{复杂度的排序}
\chapter{排序}
\section{冒泡排序}
\section{插入排序}
\section{选择排序}
\section{快速排序}
\chapter{递归}
\section{递归是什么}
\section{函数自我调用}
\section{应用场景}
\section{利用备忘录优化递归算法}
\chapter{贪心}
\section{贪心是什么}
\section{相关实例}
\section{贪心与动态规划}
\chapter{动态规划}
\section{01背包}
\section{最长公共子序列}
\section{最长公共子串}
\chapter{人工智能算法KNN}
\indent \verb|KNN|是一个人工智能算法,可以使用它计算相似度。


这一章需要读者具备初中数学能力,里面会出现一些公式,不过不用慌张,请轻松的看完。我也会在
这里给出\verb|KNN|工具包,在项目中使用即可
\section{欧几里得距离公式}
\section{余弦相似度}
\section{实例成绩预测器}
\subsection{模型设计}
\subsection{算法描述}
\subsection{代码实现}
\part{工程应用}
\chapter{网络框架}
\section{Django与Flask}
\section{Flask的下载}
\section{Flask的应用}
\chapter{网络爬虫}
\section{网络爬虫的简介}
\section{网络爬虫的库requests}
\section{源代码解析库re}
\section{反爬方法UserAgent}
\section{更多领域}
\chapter{科学计算}
\section{matplotlib与numpy}
\section{numpy}
\section{使用matplotlib画图}
\chapter{项目制作}
\section{Pypi包制作}
\section{利用资源}
\section{PyInstaller发布}

\end{document}